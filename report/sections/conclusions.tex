\section{Conclusions}
The results of this study provide a detailed comparison between a traditional TCP+TLS file transfer solution and a modern QUIC-based implementation. Our experiments, conducted under loopback conditions using a 1GB file, showed that the TCP+TLS implementation exhibits extremely low handshake latency and round-trip times, which is expected in a loopback environment. When evaluating file transfers, this solution achieved an high throughput, with upload and download speeds reaching over 280 MB/s. In contrast, the QUIC implementation demonstrates higher handshake (0.019117 s) and RTT (average 0.005623 s) values. Although these numbers are still relatively low in absolute terms, they are still worse than those observed with TCP+TLS. The biggest difference was found in the throughput values, which in QUIC's case were in the low single-digit MB/s range (approximately 1.15 MB/s for both uploads and downloads). This contrast in overall performances indicates a substantial performance bottleneck in the current QUIC setup, possibly due to its immature implementation in Python and the additional overhead of handling multiplexed streams asynchronously.