\section{Introduction}
In a time dominated by data-driven applications, efficient and secure file transfer protocols are critical for enabling real-time communication, cloud services, and large-scale data sharing. Traditional protocols like TCP (Transmission Control Protocol) paired with TLS (Transport Layer Security) have long served as the backbone for reliable and encrypted data transmission. However, emerging technologies such as QUIC (Quick UDP Internet Connections) promise to address long standing limitations of TCP, such as latency during connection setup and head-of-line blocking, while integrating modern security features like mandatory encryption.

\subsection{Motivation}
While TCP + TLS remains widely adopted, its performance bottlenecks in high-latency or unstable networks have spurred interest in alternatives. QUIC, built on top of UDP, offers features like 0-RTT connection resumption, multiplexed streams, and built-in encryption with TLS 1.3, positioning it as a compelling candidate for modern applications. However, its trade-offs in terms of resource usage, compatibility, and security robustness are less understood particularly in the context of file transfer systems. This work seeks to evaluate whether QUIC's theoretical advantages translate into practical benefits for file transfer applications compared to the well-established TCP+TLS stack.