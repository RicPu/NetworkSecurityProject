\section{Introduction}

\subsection{Background}

% Transmission Control Protocol (TCP)
The Transmission Control Protocol (TCP) is a core component of the Internet protocol suite, complementing the Internet Protocol (IP) in its original implementation. TCP ensures reliable, ordered, and error-checked delivery of a byte stream between applications communicating over an IP network. As a connection-oriented protocol, TCP requires the sender and receiver to establish a connection through a three-way handshake before communication begins. The server must be actively listening for client requests. While features like handshaking, retransmission, and error detection enhance reliability, they also increase latency. Applications prioritizing speed over reliability often use the connectionless User Datagram Protocol (UDP) instead. Operating at the transport layer, TCP provides host-to-host connectivity and abstracts the complexities of data transmission, such as IP fragmentation and maximum transmission unit considerations. It handles all handshake and transmission processes, presenting applications with a simplified network connection via a socket interface.

\subsection{Objectives}

\subsection{Scope}